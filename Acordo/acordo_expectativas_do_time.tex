%% LyX 2.3.2 created this file.  For more info, see http://www.lyx.org/.
%% Do not edit unless you really know what you are doing.
\documentclass[brazil]{article}
\usepackage[brazilian]{babel}
\usepackage[utf8]{inputenc}
\usepackage[T1]{fontenc}
\usepackage[latin9]{inputenc}
\usepackage{geometry}
\geometry{verbose,tmargin=2cm,bmargin=2cm,lmargin=2cm,rmargin=2cm}
\setcounter{secnumdepth}{2}
\setcounter{tocdepth}{2}
\usepackage{array}
\usepackage{multirow}
\usepackage{amsmath}
\usepackage{amsthm}
\usepackage{graphicx}
\usepackage{setspace}
\onehalfspacing

\makeatletter

%%%%%%%%%%%%%%%%%%%%%%%%%%%%%% LyX specific LaTeX commands.
%% Because html converters don't know tabularnewline
\providecommand{\tabularnewline}{\\}

%%%%%%%%%%%%%%%%%%%%%%%%%%%%%% Textclass specific LaTeX commands.
\numberwithin{equation}{section}
\numberwithin{figure}{section}

%%%%%%%%%%%%%%%%%%%%%%%%%%%%%% User specified LaTeX commands.
\renewcommand{\tabcolsep}{0cm}

\AtBeginDocument{
  \def\labelitemi{\normalfont\bfseries{--}}
}

\makeatother

\usepackage{babel}
\begin{document}
% \begin{tabular}{cl}
% \multirow{5}{*}{\includegraphics[width=0.15\textwidth]{\string"LogoEE1\string.png}\quad{}} & Universidade Federal de Minas Gerais\tabularnewline
%  & Escola de Engenharia\tabularnewline
%  & Departamento de Engenharia Elétrica\tabularnewline
%  & Laboratório de Projeto III\tabularnewline
%  & Prof. Bruno Vilhena Adorno\tabularnewline
% \end{tabular}

\vspace{0.5cm}

\noindent \begin{center}
{\huge{}Acordo sobre as expectativas do time}{\huge\par}
\par\end{center}

\noindent Nome dos integrantes do time: 

\noindent 1. Bruna Silva Queiroz \hrulefill{}

\noindent 2. Matheus Silva Araujo \hrulefill{}

\noindent 3. Philippe Garandy Moreira \hrulefill{}

\noindent \textbf{Rotatividade}

\begin{itemize}
    \item Iremos seguir a ordem alfabética dos papéis na atribuição das atividades: Coordenador, Relator, Verificador, Coordenador.
    \item Na primeira semana, os papéis serão: Bruna, Coordenadora; Matheus, Relator e Philippe, Verificador.
    
\end{itemize}

\noindent \textbf{Cooperações}
\begin{itemize}
    \item Vamos seguir as orientações do professor quanto a membros não-cooperativos.
    \item Nos comprometemos a ser francos com relação a eventuais dificuldades e emergências que comprometam o trabalho.
    \item Nos comprometemos também a ser solícitos para absorver os impactos de impedimentos dos outros colegas.
\end{itemize}

\noindent \textbf{Organização das atividades - processo}
\begin{itemize}
    \item Vamos seguir uma adaptação da metodologia Scrum.
    \item Vamos dividir o trabalho em Sprints.
    \begin{itemize}
    \item Para a próxima aula deveremos ter um planejamento inicial das sprints.
    \end{itemize}
    \item Teremos os ritos de \textit{daily}, \textit{sprint planning} e \textit{sprint review}.
\end{itemize}

\noindent \textbf{Horário das Reuniões}
\begin{itemize}
    \item Vamos utilizar o tempo disponível da aula, 19:00 às 22:35, para realizar todo o trabalho possível e nos preparar para as próximas atividades.
    \item Ao final de cada aula, 22:00, iremos analisar a situação real das atividades e definir os próximos passos através de uma \textit{daily}.
    \item O resultado da \textit{daily} será o relatório semanal enviado para o professor.
    \item Caso haja a necessidade de um segundo encontro, ele deverá acontecer preferencialmente aos domingos.
\end{itemize}

\noindent \textbf{Atividades preliminares}
\begin{itemize}
    \item Vamos utilizar a \textit{sprint planning} para nos prepararmos para as atividades que serão desenvolvidas.
    \item Nos comprometemos a durante a \textit{planning} avaliar a capacidade do time de cumprir as atividades propostas e caso notemos não ser possível, procurarmos alternativas que viabilizem a execução das atividades.
    \item Nos comprometemos também a seguir o planejamento definido em conjunto durante a \textit{planning}.
\end{itemize}

\noindent \textbf{Avaliações}
\begin{itemize}
    \item Nos comprometemos a acompanhar as entregas e resultados das atividades independente do papel de cada integrante na semana.
    \item Caso haja divergências quanto às notas, o grupo deve se reunir para analisar a causa raiz e utilizar este aprendizado para evitar erros futuros. 
\end{itemize}

\noindent \textbf{Comunicação e Compartilhamento}
\begin{itemize}
    \item Vamos utilizar um grupo no \textit{Whatsapp} para comunicação rápida entre os membros do time.
    \item Vamos utilizar o \textit{Trello} para gestão das atividades.
    \item Vamos utilizar o \textit{GitHub} para gestão dos artefatos de código fonte produzidos pelo time.
\end{itemize}

\end{document}
